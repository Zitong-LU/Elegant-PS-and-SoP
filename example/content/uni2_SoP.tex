\SetDocTitle{Statement of Purpose}  
% Remember to set the names of the program and the university. 
\SetProgramName{PhD Program in Eating}         % Program you're applying for
\SetUniversityName{Atlantic Institute of Culinary Sciences} % University name
\SetUniversityAbbr{AICS}                        % University abbreviation (default as the university name if not set)

Growing up along the coral-lined avenues of the Great Reef, I developed an early fascination with the act of eating—not merely as a biological necessity, but as a profound expression of identity, emotion, and community. My years as a professional food taster at the Coral Café refined my palate and deepened my understanding of how flavor, texture, and context interact to shape human experience \cite{ref1, ref2, ref3}. Yet, as I savored countless dishes, I realized that the phenomena of eating and drinking remain underexplored as scientific frontiers. This recognition fuels my desire to join the \GetProgramName{} at the \GetUniversityName{}, where I hope to unite sensory science, behavioral research, and environmental awareness to advance our understanding of consumption in both theory and practice.

\textbf{\underline{Exploring the Science of Taste and Behavior.}}
During my tenure at the Coral Café, I approached tasting not as passive evaluation but as systematic experimentation. I collaborated with chefs and marine nutritionists to examine how salinity, temperature, and ingredient composition alter sensory perception \cite{ref1}. My analysis of beverage pairings, for instance, demonstrated that minor variations in bubble density within kelp-based sodas significantly affected perceived umami intensity. Such findings inspired me to formalize my curiosity through structured research.
I also spearheaded a project to quantify “gustatory satisfaction” among diners using a combination of psychometric surveys and physiological indicators. The initiative revealed meaningful associations between emotional states and food intake rhythms, highlighting the psychological depth of eating behaviors. These experiences have shown me that eating and drinking, though everyday acts, are complex scientific phenomena deserving rigorous inquiry. A doctoral program will provide the framework to refine these investigations and translate sensory experience into measurable knowledge.

\textbf{\underline{Why Especially \GetUniversityAbbr{}?}}
The \GetUniversityName{} stands out as a global leader in the empirical study of gastronomy, taste, and consumption. Its Sensory Dynamics Laboratory, renowned for integrating neurogastronomy with environmental nutrition, offers an unparalleled environment for the kind of interdisciplinary work I seek to pursue. The institute’s ethos—melding culinary artistry with scientific precision—resonates deeply with my belief that meaningful innovation occurs when disciplines converge.
Equally compelling is the opportunity to collaborate with Professor Marina Salt, whose research on adaptive marine diets and sustainable beverage systems has profoundly influenced my intellectual development. Her investigations into fluid–solid interaction dynamics mirror my own interests in multisensory integration and ecological eating practices. Working alongside her would not only refine my research methodology but also situate my work within a broader conversation on sustainability and sensory science.
Through the \GetProgramName{}, I aspire to advance the understanding of how humans and marine species experience, interpret, and adapt their eating and drinking behaviors in changing environments. My long-term goal is to establish a Laboratory for Integrated Consumption Studies, devoted to promoting sustainable, evidence-based approaches to nourishment and pleasure. With its interdisciplinary culture, visionary faculty, and commitment to scientific excellence, the \GetUniversityName{} provides the ideal foundation upon which to build this intellectual and practical enterprise.